\usepackage{booktabs}
\usepackage{amsthm}
\usepackage[pagebackref=false,breaklinks=true,letterpaper=true,colorlinks,bookmarks=false]{hyperref}
\usepackage{tcolorbox}
\usepackage{color}
\usepackage{framed}
\setlength{\fboxsep}{.8em}

\usepackage{fontspec} % Font package

% Select fonts
%\setmainfont{Times New Roman}
%\setsansfont{Arial}

%
% You must use XELATEX for the font stuff to work.
%
% Also notes for Windows:
%
% Install the Iosevka fonts in the `C:/texlive/2024/texmf-dist/fonts/truetype/public/` fonts directory where the others already are.) 
% It can be obtained at https://github.com/be5invis/Iosevka -> Releases.
%
% Run `fc-cache -f --verbose` to refresh the XeTeX / fontconfig cache (as seen at https://github.com/NixOS/nixpkgs/issues/24485, but this works for a TeX install on MSWindows too, fortunately)
%
% WARNING: ^^^ execute the above command in an ADMINISTRATOR command shell or you will be treated to all sorts of fontconfig crashes and "Fontconfig error: No writable cache directories"!
%
% Check the Iosevka font is reachable by inspecting the output of this command: `fc-check Iosevka`, which must say something like this:
%
% ```
% Iosevka-Regular.ttc: "Iosevka" "Regular"
% ```
%
% NOTE that before I got the fontcache to update properly, that same `fc-check Iosevka` command produced this nonsense:
% ```
% CascadiaCode-Regular.otf: "Cascadia Code" "Regular"
% ```
% so be careful to READ what is output as the errors (like this one) are very much non-obvious.
%
% Also note that dropping the fonts in the Windows Fonts directory didn't seem to produce the desired result (but then I still was struggling with fc-cache so that may be the important factor),
% also note that editing the `Path` option in the `\setmonofont` statement below didn't deliver either as it all seems to always harken back to the (headache) fontconfig configuration of your local TeX rig.
%

%
% Incidentally, when you're fiddling with this preamble stuff, it's much faster to just tweak the generated `.tex` file until you're satisfied.
% The command to produce a PDF file straight off that one is:
%
%      xelatex -no-shell-escape probriskreward-bookdown.tex
%

% Iosevka-Regular.ttc, Iosevka-*.ttc
\setmonofont{Iosevka}[
  % Color = {FF1934} ,
  % Path = {./PkgTTC-Iosevka/} ,
  % Extension = .ttc ,
  % UprightFont = *-Light ,
  % BoldFont = *-Regular ,
  % FontFace = {k}{n}{*-Heavy} ,
  Scale = MatchLowercase ,
  % Scale = 0.85 ,
  % Ligatures = TeX ,
  % Contextuals = {Alternate} ,
]
% \setmonofont[ItalicFont={Consolas Italic},BoldFont={Consolas Bold},BoldItalicFont={Consolas Bold Italic}]{Consolas}

\makeatletter
\def\thm@space@setup{%
  \thm@preskip=8pt plus 2pt minus 4pt
  \thm@postskip=\thm@preskip
}
\makeatother


\newtcolorbox{examplebox}{
  colback=green,
  colframe=orange,
  coltext=black,
  boxsep=5pt,
  arc=4pt}

\newtcolorbox{theorembox}{
  colback=green,
  colframe=green,
  coltext=black,
  boxsep=5pt,
  arc=4pt}

\newtcolorbox{definitionbox}{
colback=white,
colframe=green,
coltext=black,
boxsep=5pt,
arc=4pt}